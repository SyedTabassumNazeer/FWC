\documentclass{article}
\usepackage{setspace}
\usepackage{gensymb}
\usepackage{xcolor}
\usepackage{caption}
\usepackage[hyphens,spaces,obeyspaces]{url}
%\usepackage{subcaption}
%\doublespacing
\singlespacing
%\usepackage{graphicx}
%\usepackage{amssymb}
%\usepackage{relsize}
\usepackage[cmex10]{amsmath}
\usepackage{mathtools}
%\usepackage{amsthm}
%\interdisplaylinepenalty=2500
%\savesymbol{iint}
%\usepackage{txfonts}
%\restoresymbol{TXF}{iint}
%\usepackage{wasysym}
\usepackage{amsthm}
\usepackage{mathrsfs}
\usepackage{txfonts}
\usepackage{stfloats}
\usepackage{cite}
\usepackage{cases}
\usepackage{subfig}
%\usepackage{xtab}
\usepackage{longtable}
\usepackage{multirow}
%\usepackage{algorithm}
%\usepackage{algpseudocode}
\usepackage{enumerate}
\usepackage{mathtools}
\usepackage{eenrc}
%\usepackage[framemethod=tikz]{mdframed}
%\usepackage{breakcites}
\usepackage{listings}
    \usepackage[latin1]{inputenc}                                 %%
    \usepackage{color}                                            %%
    \usepackage{array}                                            %%
    \usepackage{longtable}                                        %%
    \usepackage{calc}                                             %%
    \usepackage{multirow}                                         %%
    \usepackage{hhline}                                           %%
    \usepackage{ifthen}                                           %%
  %optionally (for landscape tables embedded in another document): %%
    \usepackage{lscape}     

\usepackage{tikz}
\usepackage{circuitikz}
\usepackage{karnaugh-map}
\usepackage{pgf}
\usepackage[hyphenbreaks]{breakurl}
\maketitle
\tableofcontents
\bigskip
\renewcommand{\thefigure}{\theenumi}
\renewcommand{\thetable}{\theenumi}

%\usepackage{url}
%\def\UrlBreaks{\do\/\do-}





%\usepackage{stmaryrd}


%\usepackage{wasysym}
%\newcounter{MYtempeqncnt}
\DeclareMathOperator*{\Res}{Res}
%\renewcommand{\baselinestretch}{2}
\renewcommand\thesection{\arabic{section}}
\renewcommand\thesubsection{\thesection.\arabic{subsection}}
\renewcommand\thesubsubsection{\thesubsection.\arabic{subsubsection}}

\renewcommand\thesectiondis{\arabic{section}}
\renewcommand\thesubsectiondis{\thesectiondis.\arabic{subsection}}
\renewcommand\thesubsubsectiondis{\thesubsectiondis.\arabic{subsubsection}}

% correct bad hyphenation here
\hyphenation{op-tical net-works semi-conduc-tor}

%\lstset{
%language=C,
%frame=single, 
%breaklines=true
%}

%\lstset{
	%%basicstyle=\small\ttfamily\bfseries,
	%%numberstyle=\small\ttfamily,
	%language=Octave,
	%backgroundcolor=\color{white},
	%%frame=single,
	%%keywordstyle=\bfseries,
	%%breaklines=true,
	%%showstringspaces=false,
	%%xleftmargin=-10mm,
	%%aboveskip=-1mm,
	%%belowskip=0mm
%}

%\surroundwithmdframed[width=\columnwidth]{lstlisting}
\def\inputGnumericTable{}                                 %%
\lstset{
%language=C,
frame=single, 
breaklines=true,
columns=fullflexible
}
 

\begin{document}
%

\theoremstyle{definition}
\newtheorem{theorem}{Theorem}[section]
\newtheorem{problem}{Problem}
\newtheorem{proposition}{Proposition}[section]
\newtheorem{lemma}{Lemma}[section]
\newtheorem{corollary}[theorem]{Corollary}
\newtheorem{example}{Example}[section]
\newtheorem{definition}{Definition}[section]
%\newtheorem{algorithm}{Algorithm}[section]
%\newtheorem{cor}{Corollary}
\newcommand{\BEQA}{\begin{eqnarray}}
\newcommand{\EEQA}{\end{eqnarray}}
\newcommand{\define}{\stackrel{\triangle}{=}}
\bibliographystyle{IEEEtran}
%\bibliographystyle{ieeetr}
\providecommand{\nCr}[2]{\,^{#1}C_{#2}} % nCr
\providecommand{\nPr}[2]{\,^{#1}P_{#2}} % nPr
\providecommand{\mbf}{\mathbf}
\providecommand{\pr}[1]{\ensuremath{\Pr\left(#1\right)}}
\providecommand{\qfunc}[1]{\ensuremath{Q\left(#1\right)}}
\providecommand{\sbrak}[1]{\ensuremath{{}\left[#1\right]}}
\providecommand{\lsbrak}[1]{\ensuremath{{}\left[#1\right.}}
\providecommand{\rsbrak}[1]{\ensuremath{{}\left.#1\right]}}
\providecommand{\brak}[1]{\ensuremath{\left(#1\right)}}
\providecommand{\lbrak}[1]{\ensuremath{\left(#1\right.}}
\providecommand{\rbrak}[1]{\ensuremath{\left.#1\right)}}
\providecommand{\cbrak}[1]{\ensuremath{\left\{#1\right\}}}
\providecommand{\lcbrak}[1]{\ensuremath{\left\{#1\right.}}
\providecommand{\rcbrak}[1]{\ensuremath{\left.#1\right\}}}
\theoremstyle{remark}
\newtheorem{rem}{Remark}
\newcommand{\sgn}{\mathop{\mathrm{sgn}}}
\providecommand{\abs}[1]{\left\vert#1\right\vert}
\providecommand{\res}[1]{\Res\displaylimits_{#1}} 
\providecommand{\norm}[1]{\lVert#1\rVert}
\providecommand{\mtx}[1]{\mathbf{#1}}
\providecommand{\mean}[1]{E\left[ #1 \right]}
\providecommand{\fourier}{\overset{\mathcal{F}}{ \rightleftharpoons}}
%\providecommand{\hilbert}{\overset{\mathcal{H}}{ \rightleftharpoons}}
\providecommand{\system}{\overset{\mathcal{H}}{ \longleftrightarrow}}
	%\newcommand{\solution}[2]{\textbf{Solution:}{#1}}
\newcommand{\solution}{\noindent \textbf{Solution: }}
\providecommand{\dec}[2]{\ensuremath{\overset{#1}{\underset{#2}{\gtrless}}}}
%\numberwithin{equation}{subsection}
\numberwithin{equation}{section}
%\numberwithin{problem}{subsection}
%\numberwithin{definition}{subsection}
\makeatletter
\@addtoreset{figure}{problem}
\makeatother
\let\StandardTheFigure\thefigure
%\renewcommand{\thefigure}{\theproblem.\arabic{figure}}
\renewcommand{\thefigure}{\theproblem}
%\numberwithin{figure}{subsection}
%\numberwithin{equation}{subsection}
%\numberwithin{equation}{section}
%%\numberwithin{equation}{problem}
%%\numberwithin{problem}{subsection}
\numberwithin{problem}{section}
%%\numberwithin{definition}{subsection}
%\makeatletter
%\@addtoreset{figure}{problem}
%\makeatother
\makeatletter
\@addtoreset{table}{problem}
\makeatother
\let\StandardTheFigure\thefigure
\let\StandardTheTable\thetable
%%\renewcommand{\thefigure}{\theproblem.\arabic{figure}}
%\renewcommand{\thefigure}{\theproblem}
\renewcommand{\thetable}{\theproblem}
%%\numberwithin{figure}{section}
%%\numberwithin{figure}{subsection}
\vspace{3cm}
%\documentclass{article}
\usepackage[utf8]{inputenc}
\usepackage{tikz}
\usepackage{karnaugh-map}
\usepackage{graphics}
\usepackage{hyperref}
\title{\textbf{BINARY HALF SUBTRACTOR}}
\author{Vallabhaneni Rupa Sai Sreshta}
    \begin{karnaugh-map}
        \contingut{0,0,0,0,0,0,0,0,0,0,0,0,0,0,0,0}
       \implicant{0}{2}{red}
       \implicant{5}{15}{purple}
       \implicantdaltbaix[3pt]{3}{10}{blue}
    \implicantcantons[2pt]{orange}
       \implicantcostats{4}{14}{green}
    \end{karnaugh-map}
\begin{tableofcontents}
\begin{abstract}
This manual shows how to execute Binary-half-Subtractor having two inputs A and B.
\end{abstract}
\section{COMPONENTS}
\centering
\begin{tabular}{|c||c||c|}
\hline
\textbf{component} & {value} & {quantity} \\
\hline
 \textbf{Breadboard}  &  &  1  \\
 \hline
 \textbf{Resistor}  & {220ohm} & {1} \\
 \hline
 \textbf{Arduino} & {Uno} & 1\\
 \hline
 \textbf{Sevensegment display} & {Common anode} & {1}\\
 \hline
 \textbf{Jumper wires} &   &  {20}\\
 \hline
%\centering{\textbf{Table}}
\\
\section{BINARY HALF SUBTRACTOR WITH TWO INPUTS}
Half subtractor is a combination circuit with two inputs and two outputs which is difference and borrow. It produces the difference between the two binary bits at the input and also produces an output (Borrow) to indicate if a 1 has been borrowed. In the subtraction (A-B), A is called a Minuend bit and B is called as Subtrahend bit.
\newline
\section{TRUTH TABLE}
\begin{kvmap}
\kvlist{4}{4}{0,1,1,0,0,1,0,1,0,1,1,1,0,0,1,1}{a,b,c ⌋
,d,e,f}
\bundle[color=red]{1}{1}{1}{2}
\bundle{3}{2}{2}{3}
\bundle[color=blue,reducespace=3pt]{3}{2}{3}{1}
\end{kvmap}
\vspace{7mm}
\begin{tabular}{|c||c||c||c|}
\hline
\textbf{A} & {B} & {D} & {X}\\
\hline
\textbf{0} & {0} & {0} & {0}\\
\textbf{0} & {1} & {1} & {1}\\
\textbf{1} & {0} & {1} & {0}\\
\textbf{1} & {1} & {0} & {0}\\
\hline
\end{tabular}
\newline
\newline
\newline
\textbf{Equations :-}
\newline
\newline
Equation for D is
\begin{equation}
D= A'B+AB'
\end{equation}
Equation for X is
\begin{equation}
X=A'B
\end{equation}
\\
\begin{tabular}{|c|c|c|}
\hline
\textbf{Arduino} & 2 & 3  \\
\hline
\textbf{Display} & {a} & {b}  \\
\hline
\end{tabular}
\\
\centering
\textbf{Table-1}
\centering
\\
\textbf{Problem:}
Make the connections as per Table-1 and execute the following program.
\end{tableofcontents}
\\

\end{document}